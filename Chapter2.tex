\chapter{Heavy Semileptonic Decays}
\label{chap:semileptonic}

\section{Strong Interaction Physics}
\subsection{Quantum Chromodynamics}
\subsection{Chiral Perturbation Theory}
\section{Heavy Quark Physics}
\subsection{HQET}
\subsection{NRQCD}

\section{Form Factors}

We now turn to some more technical details about semileptonic decays.
\\ \\
We will refer to the 4-momenta and mass of the initial and final states as $p_{1,2},M_{1,2}$ respectively, and define the 4-momentum taken away by the $W^{\pm}$ boson as $q \equiv p_2 - p_1$. We work in the rest frame of the initial meson, in which
\begin{align}
	q^2 = M_1^2 + M_2^2 - 2M_1 E_2.
\end{align}
There is a physically allowed range of values for an on-shell $q^2$. The minimum is when all of the 3-momentum of the initial state is taken by the final meson, $q_{\text{min}} ^2 = 0$. $q^2$ is maximised when all of the 3-momentum is taken by the boson, $\underline{p}_2^2 = 0 \rightarrow E_2 = \sqrt{M_2^2 + \underline{p}_2^2 } = M_2 \rightarrow q_{\text{max}}^2 = M_1^2 - M_2^2 - 2M_1M_2 = ( M_2 - M_1 )^2$. So
\begin{align}
	0 \leq q^2 \leq ( M_2 - M_1 )^2
\end{align}
This also creates an allowed range for the final meson 3-momentum:
\begin{align}
	0 < \underline{p}_{2}^2 < \left( M_1^2 + M_2^2 \over 2M_1 \right)^2 - M_2^2
\end{align}
\\ \\
Hadronic matrix elements like $H_{\mu}$ from sec. \ref{sec:cp} can be parameterised in terms of form factors. The current operator between between the states is a conserved current, so then the matrix element must be proportional to only conserved quantities, namely, elements of the stress-energy tensor. Lorentz invariance requires indices on either side of such a relation match, so a matrix element with a single Lorentz index can only be proportional to 4-momenta.
\\ \\
Let us consider the case where the two states in the matrix element are pseudoscalar mesons (as in $B_s\to D_s l\nu$), with an insertion of a left-handed current (i.e. the coupling to the $W^{\pm}$ in \eqref{eq:weakL}), $J_{\mu}^{ij} = \bar{u}_i \gamma_{\mu} {1\over2}(1-\gamma_5) d_j$. This can be written as $J^{ij}_{\mu} = V^{ij}_{\mu} - A^{ij}_{\mu}$, these are the vector and axial vector currents. 
\\ \\
The axial vector evaluated between two pseudoscalar states must vanish because the combination is not parity invariant thus does not contribute in pure QCD, leaving just the vector current. The most common parameterisation is:
\begin{align}
\label{eq:formfactors}
\langle P_2 (p_2) | V^{\mu} | P_1 (p_1) \rangle  &= f_{+} (q^2) \left[ p_1^{\mu} + p_2^{\mu} - {M_1^2 - M_2^2 \over q^2} q^{\mu} \right] + f_0(q^2) {M_1^2 - M_2^2 \over q^2 } q^{\mu}
\end{align}
Where $|P_i(p_i)\rangle$ is a pseudoscalar meson state with momentum $p_i$.

\subsection{Analyticity}
\subsection{z-Expansion}


\section{Renormalization of Currents}


The currents $V_{\mu}$ and $A_{\mu}$ are conserved in the Chiral limit (quark masses $\rightarrow 0$). This can be shown via Noether's theorem: consider a theory containing a number of fields $\{\varphi_i\}$, which transform under a group $G$ by
\begin{align}
	\varphi_i \rightarrow \varphi_i (x) - i\epsilon_a (x) F_{i,a}[\{\varphi\}]
\end{align}
It can be shown that there exists currents
\begin{align}
	J_a^{\mu} = -i{\partial \mathscr{L} \over \partial\partial_{\mu} \varphi_i} F_{i,a}
\end{align}
which obey
\begin{align}
	\label{eq:conservedJ1}
	J^{\mu}_a = {\partial \delta \mathscr{L} \over \partial \partial_{\mu} \epsilon_a } \\
	\partial_{\mu} J^{\mu}_a = {\partial \delta \mathscr{L} \over \delta \epsilon_a}	
	\label{eq:conservedJ2}
\end{align}
where $\delta\mathscr{L}$ is the result of an infinitesimal $G$ operation, i.e. $G$ : $\mathscr{L} \rightarrow \mathscr{L} + \delta\mathscr{L}$. From \eqref{eq:conservedJ2}, we see that if the theory is symmetric under the generator parameterised by $\epsilon_a$, then $J_a^{\mu}$ is a conserved current.
\\ \\
In the chiral limit, QCD with $N$ flavors has the accidental global symmetry ("chiral symmetry") $U(N)_L\times U(N)_R$:
\begin{align}
	q_{L/R} \rightarrow \text{exp}(-i\theta_a^{L/R} \lambda_a) q_{L/R}
	\label{eq:LRtransform}
\end{align} 
where $q$ is a vector in flavor space, and $\lambda_a$ are the generators of $U(N)$ in the fundemental representation and acts on flavor. Applying \eqref{eq:conservedJ1} and \eqref{eq:conservedJ2} to QCD we get the conserved currents:
\begin{align}
	J_{L/R}^{\mu,a} &= \bar{q}_{L/R} \gamma^{\mu} \lambda_a q_{L/R} \\
	\partial_{\mu} J_{L/R}^{\mu,a} &= 0
\end{align}
$J_{L/R}$ are often expressed instead in terms of {\it{vector}} and {\it{axial}} currents
\begin{align}
	V^{\mu,a} &= J^{\mu,a}_L + J^{\mu,a}_R = \bar{q} \gamma^{\mu} \lambda_a q \\
	A^{\mu,a} &= J^{\mu,a}_L - J^{\mu,a}_R = \bar{q} \gamma^{\mu} \gamma_5 \lambda_a q
\end{align}
which are conserved currents corresponding to the vector and axial symmetries $U(N)_V$ and $U(N)_A$, consisting of simultanious $L/R$ transformations above with constraints $\theta_a^{L} = \pm \theta_a^{R}$. Vector and axial currents between any two individal flavours, i.e.
\begin{align}
	\label{eq:individualflavs1}
	V^{\mu}_{ij} = \bar{q_i} \gamma^{\mu} q_j \\
	A^{\mu}_{ij} = \bar{q_i} \gamma^{\mu} \gamma_5 q_j
	\label{eq:individualflavs2}
\end{align}
are also conserved, since they can be constructed from linear combinations of $V^{\mu,a}$ and $A^{\mu,a}$.
\\ \\
$U(N)$ can be broken up into $SU(N)\times U(1)$, where $U(1)$ is a singlet transformation, i.e., of the form of \eqref{eq:LRtransform} with $\lambda = 1$. When QCD is quantized, it develops an anomaly which breaks the singlet axial symmetry. This reduces the symmetry group to $SU(N)_V\times SU(N)_A\times U(1)_V$, and prevents the corresponding axial singlet current $A^{\mu,0}$ from being conserved.
\\ \\
In the case of non-zero quark mass, the chiral symmetry is broken. This leads to \eqref{eq:conservedJ2} for the vector and axial currents evaluating instead as
\begin{align}
	\partial_{\mu} V^{\mu,a} &= i\bar{q} [ M,  \lambda_a ] q \\
	\partial_{\mu} A^{\mu,a} &= i\bar{q} \{ M, \lambda_a \} q 
\end{align}
where $M$ is the quark mass matrix in flavor space. These are the parially conserved axial and vector current identities (PCAC and PCVC). By taking a linear combination of these equations, one can derive PCAC and PCVC for individual flavors:
\begin{align}
	\partial_{\mu} V_{ij}^{\mu} &= i(m_i - m_j) \bar{q_i} q_j \equiv i(m_i - m_j) S_{ab} \\
	\partial_{\mu} A_{ij}^{\mu} &= i(m_i - m_j) \bar{q_i} \gamma_t q_j \equiv i(m_i - m_j) P_{ij}
\end{align}
We refer to $S$ as the scalar current and $P$ as the pseudoscalar. These identities translates straightforwardly to expectation values:
\begin{align}
	\label{eq:PCVC}
	(p_1 - p_2)_{\mu} \langle \psi(p_2) | V^{\mu}_{ij} | \phi(p_1) \rangle &= (m_i - m_j) \langle \psi(p_2) | S_{ij} | \phi(p_1) \rangle \\
	(p_1 - p_2)_{\mu} \langle \psi(p_2) | A^{\mu}_{ij} | \phi(p_1) \rangle &= (m_i - m_j) \langle \psi(p_2) | P_{ij} | \phi(p_1) \rangle
\end{align}
where $| \psi(p) \rangle$ and $| \phi(p) \rangle$ are arbitrary states containing momentum $p$.
\\ \\
These are powerful tools for connecting expectation values of different currents. For example, by combining \eqref{eq:PCVC} and \eqref{eq:formfactors}, we see that:
\begin{align}
	\langle P_2(p_2) | S_{ij} | P_1(p_1) \rangle = {M_1^2 - M_2^2 \over m_i - m_j} f_0(q^2)
\end{align}
Hence we have an extra route to accessing $f_0$. In a lattice simulation which computes expectation values of $V^{\mu}$, one could constrain $f_0$ by also computing the scalar current on the lattice, then use the rest of the information in $V^{\mu}$ to constrain $f_+$.