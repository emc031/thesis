\chapter{Heavy Semileptonic Decays}
\label{chap:semileptonic}

\section{Strong Interaction Physics}
\subsection{Quantum Chromodynamics}
\subsection{Chiral Perturbation Theory}
\section{Heavy Quark Physics}

some stuff about the different scales in a $B$ meson, analogy to hydrogen, etc.

\subsection{HQET}

Heavy Quark Effective Theory (HQET) is an effective field theory with the cutoff at the heavy quark mass $m_Q$, and terms organized in a series in $\Lambda_{\text{QCD}}/m_Q$. Since at the $b$ (and $c$) mass QCD is perturbative ($\alpha_s(m_Q) << 1$), one can match HQET to perturbative QCD at $m_Q$, then run the couplings of HQET down to produce useful predictions at the confinement scale.

\subsubsection{HQET Lagrangian}

As a simple example, we will derive HQET for a single heavy quark interacting with gluons. The fermion part of the Lagrangian is
\begin{align}
	\mathscr{L}_{\text{QCD}} = \bar{Q} ( i\slashed{D} - m_Q ) Q,
\end{align}
where $Q$ is the heavy quark field. Define the heavy quark velocity $v$ according to
\begin{align}
	v = {p_Q \over m_Q}.
\end{align}
Now split $Q$ into "heavy" and "light" components:
\begin{align}
	\label{eq:hdef}
	Q = h + H\quad : \quad &h = {1\over 2} e^{-im_Q v\cdot x} ( 1 + \slashed{v} ) Q \\
	&H = {1\over 2} e^{-im_Q v\cdot x} ( 1 - \slashed{v} ) Q
\end{align}
with the important property
\begin{align}
	\slashed{v} h = h \quad \slashed{v} H = - H.
\end{align}
In terms of these new fields the Lagrangian becomes
\begin{align}
	\mathscr{L}_{\text{QCD}} = i\bar{h} (v\cdot D) h - \bar{H} ( i(v\cdot D) - 2m_Q ) H 
		+ i \bar{h} \slashed{D}^{\perp} H + i \bar{H} \slashed{D}^{\perp} h.
		\label{eq:HQET_preintegral}
\end{align}
where $v_{\mu}(v\cdot D)$ is the covariant derivative projected along the direction of $v$, and $D^{\perp} = D - v_{\mu}(v\cdot D)$ is the components perpendicular to $v$. In the rest frame of the heavy quark, $v = (1,0,0,0)$ so $v_{\mu}(v\cdot D)$ becomes the temporal derivative and $D^{\perp}$ the spacial. 
\\ \\
From \eqref{eq:HQET_preintegral}, we see that $h$ is a massless field and $H$ has a mass of $2m_Q$. From this Lagrangian we can derive an equation of motion for $H$:
\begin{align}
	( i(v\cdot D) + 2m_Q) H = i\slashed{D}^{\perp} h,
\end{align}
with the solution
\begin{align}
	H = {1\over i(v\cdot D) + 2m_Q} i\slashed{D}^{\perp} h = {1\over 2m_Q}\sum_{n=0}^{\infty} {(-i(v\cdot D))^n\over 2m_Q} \slashed{D}^{\perp} h.
\end{align}
By substituting this into the Lagrangian we arrive at
\begin{align}
	\mathscr{L}_{\text{HQET}} = i \bar{h} (v\cdot D) h - \bar{h} \slashed{D}^{\perp} {1\over 2m_Q}\sum_{n=0}^{\infty} {(-i(v\cdot D))^n\over 2m_Q} \slashed{D}^{\perp} h.
\end{align}
This can be found by a more rigorous proof by performing the Gaussian integration over the $H$ field in the path integral. Since we expect $v\cdot D \sim \Lambda_{\text{QCD}}$, we can interpret the infinite sum as a series in $\Lambda_{QCD}/m_Q$, and truncate it at some order. For example to $\mathcal{O}(\Lambda_{\text{QCD}}/m_Q)$, we have
\begin{align}
	\mathcal{L}_{\text{HQET}^1} = i \bar{h} (v\cdot D) h - {1\over 2m_Q} \bar{h} \slashed{D}^{\perp 2} h
	\label{eq:HQET1}
\end{align}
 Leading order HQET exhibits new symmetries not present in full QCD, known as the heavy quark symmetries. Since $m_Q$ is not present in the leading order Lagrangian, there is a flavour symmetry - a set of $N$ heavy quarks with the same $v$ can be mixed via an $SU(N)$ symmetry. Similarly due to the absense of spin mixing matrices, a heavy quark has an $SU(2)$ spin symmetry. This builds up a physical picture of a heavy quark in a meson being a static colour charge, the dynamics at $\Lambda_{\text{QCD}}$ is not effected by it's mass or spin. This is analagous to a proton in a hydrogen atom, which similarly can be considered a static charge, since it's mass is at a higher scale than the electrons and photons. ({\color{red}why static tho?})

\subsubsection{Isgur-Wise Function}

A consequence of heavy quark symmetry relevant to semileptonic decays are the Wigner-Eckart theorems. Consider a transition amplitude between two heavy pseudoscalar mesons:
\begin{align}
	\langle M(v) | \bar{h} \Gamma h | M(v') \rangle
\end{align}
The spin structure of $| M(v) \rangle$ is $\gamma_5 (1-\slashed{v})$, this can be shown with the following argument. The state can be generally written as $| M(v) \rangle = \int d^4x d^4y f(x,y) \bar{h}(x) \gamma_5 q(y) | \Omega \rangle$ which, using $\slashed{v} h = h$, can be reexpressed as $| M(v) \rangle = \int d^4x d^4y f(x,y) \bar{h}(x) \gamma_5 (1-\slashed{v}) q(y) | \Omega \rangle /2$. Then via the spin symmetry, one can always rotate the $h$ spin in the meson state such that it matches the spin of the current, i.e. $h_{\alpha} \bar{h}_{\beta} \to 1_{\alpha\beta} f(h,\bar{h})$. Then the amplitude can be written as
\begin{align}
	\langle M(v) | \bar{h} \Gamma h | M(v') \rangle = m_M Tr[ {1\over 2} \gamma_5 (1 - \slashed{v}) \Gamma {1\over 2} \gamma_5 (1 - \slashed{v'}) \mathcal{M}(v,v')]
	\label{eq:wignereckart1}
\end{align} 
where $\mathcal{M}(v,v')$ can be any gamma-matrix valued function. The $m_M$ factor comes from the relativistic normalization of the states. A general spin decomposition of this is
\begin{align}
	\mathcal{M}(v,v') = \xi_0(v\cdot v') + \slashed{v} \xi_1(v\cdot v') + \slashed{v'} \xi_2(v\cdot v') + \slashed{v}\slashed{v'} \xi_4(v\cdot v').
\end{align}
Plugging this into \eqref{eq:wignereckart1}, we can then write the amplitude in terms of a single function:
\begin{align}
\langle M(v) | \bar{h} \Gamma h | M(v') \rangle = m_M Tr[ {1\over 2} \gamma_5 (1 - \slashed{v}) \Gamma {1\over 2} \gamma_5 (1 - \slashed{v'}) ] \xi(v\cdot v')
\end{align}
where $\xi(v\cdot v') = \xi_0(v\cdot v') + \xi_1(v\cdot v') - \xi_3(v\cdot v') - \xi_4(v\cdot v')$ is known as the Isgur-Wise function. For a general pair of mesons with spin structure $\mathcal{H}$,$\mathcal{H}'$, a transition amplitude between them with a heavy current insertion can always be written as
\begin{align}
	\langle \mathcal{H} | \bar{h} \Gamma h | \mathcal{H}' \rangle = \xi(v\cdot v') Tr[ \bar{\mathcal{H}} \Gamma \mathcal{H} ] + \order{\Lambda_{\text{QCD}}\over m_Q}
\end{align}
So all heavy semileptonic decays involving any combination of masses or spins are described by a single non-perturbative function, $\xi(v\cdot v')$. A couple of relevant examples are:
\begin{align}
	\langle D(v') | \bar{c_{v'}} \gamma^{\mu} b_v | \bar{B}(v) \rangle = \sqrt{m_B m_D} ( v + v')^{\mu} \xi(v\cdot v') \\
	\langle D^*(v') | \bar{c_{v'}} \gamma^{\mu} b_v | \bar{B}(v) \rangle = i \sqrt{m_B m_D*} \epsilon^{\mu\nu\alpha\beta} \varepsilon_{\nu}^* v'_{\alpha} v_{\beta} \xi(v\cdot v')  \\
	\langle D^*(v') | \bar{c_{v'}} \gamma^{\mu}\gamma_5 b_v | \bar{B}(v) \rangle = \sqrt{m_B m_D*} [ \varepsilon^{*\mu}(v\cdot v' + 1) - v^{'\mu} \varepsilon^*\cdot v] \xi(v\cdot v').
\end{align}
Here we have subscripted the fields $c_{v'}$,$b_v$ to specify the velocity used to separate those fields from the heavy components e.g. in eq. \eqref{eq:hdef}. 

\subsubsection{AG and Luke's Theorem}

Luke's theorem, which can be derived from the Ademollo-Gatto (AG) theorem, tells us the leading order heavy quark mass dependance of form factors. First we will derive the AG theorem. We will follow the proof given in \cite{Lebed:1991sq}.
\\ \\
Consider the transition amplitude
\begin{align}
	\langle \alpha | Q_a | \beta \rangle
\end{align}
where $Q_a$ is a conserved charge associated with some global symmetry $\mathcal{G}$, and $|\alpha\rangle$ and $|\beta\rangle$ belong to an irrep of $\mathcal{G}$. Imagine explicitly breaking the symmetry with a term like $\mathscr{L}_{\text{break}} = \lambda \mathcal{O}_{\text{break}}$. The states in the broken theory can be expressed as
\begin{align}
	|\beta \rangle = c_{\beta\beta} | \beta' \rangle + \sum_{m} c_{\beta m} | m' \rangle \\
	\langle \alpha | = c^*_{\alpha\alpha} \langle \alpha' | + \sum_{n} c^*_{\alpha n} \langle n' |.
\end{align}
where primed states are states belonging to irreps of $\mathcal{G}$. Here $|m'\rangle$ can only be states that can be mixed with $| \beta \rangle$ by $\mathcal{O}_{\text{break}}$ via the broken dynamics of the theory, and similarly for $\langle n' |$ and $\langle \alpha |$. The transition amplitude becomes
\begin{align}
\nonumber
	\langle \alpha | Q_a | \beta \rangle
	 	&= c_{\alpha\alpha}^* c_{\beta\beta} \langle \alpha' | Q_a | \beta' \rangle \\ 
	 	\nonumber
	 	&+ \sum_m c_{\alpha\alpha}^* c_{\beta m} \langle \alpha' | Q_a | m' \rangle \\
	 	\nonumber 
	 	&+ \sum_n c_{\alpha n}^* c_{\beta \beta} \langle n' | Q_a | \beta \rangle \\ 
	 	&+ \sum_m\sum_n c_{\alpha n}^* c_{\beta m} \langle n' | Q_a | m' \rangle
	 	\label{eq:AGproofexpanded}
\end{align}
The theorem applies to the situation where $|n'\rangle$ and $|m'\rangle$ live in different $\mathcal{G}$ irreps to $|\alpha\rangle$ and $|\beta \rangle$ (we assume $|\alpha\rangle$ and $|\beta \rangle$ to be in the same irrep otherwise the transition amplitude will always be zero). In this case the amplitudes in the second and third terms vanish. Now consider the order of the coefficients $c_{nm}$. We can assume that $c_{nm} = \order{\lambda}$ for arbitrary $n,m \neq \alpha,\beta$, since switching off the symmetry breaking by setting $\lambda=0$ should cause $|\alpha\rangle $ and $|\alpha'\rangle$ to coencide. Then, using the normalization of the states $\sum_{n} |c_{\alpha n} |^2 = 1$, we find $c_{\alpha\alpha} = \sqrt{1 - \order{\lambda}^2} = 1 + \order{\lambda^2}$, and similarly for $c_{\beta\beta}$. Applying this to the two surviving terms in \eqref{eq:AGproofexpanded}, we end up with
\begin{align}
	\langle \alpha | Q_a | \beta \rangle = 1 + \order{\lambda^2}
\end{align}
This is the AG theorem: if the current $Q_a$ and the symmetry breaking therm $mathcal{O}$ act orthogonally on the states, the transition amplitude can have at most a second order correction in the symmetry breaking parameter.
\\ \\
Now we will apply this to HQET to produce Luke's theorem. Consider a transition including two heavy quarks ($b$ and $c$). Then, the heavy quark symmetry is a spin symmetry for each flavour, and a flavour symmetry between them. The leading order spin symmetry breaking terms can be found from \eqref{eq:HQET1} to be
\begin{align}
	{1\over 4m_Q} \bar{h} \gamma^{\mu} \gamma^{\nu} F_{\mu\nu} h 
\end{align}
for both $h=b$ and $h=c$. The leading order flavour breaking term is
\begin{align}
	\left({1\over 2m_b} - {1\over 2m_c} \right) {1\over 2} \bar{h} \sigma_z \slashed{D}^{\perp 2} h
\end{align}
where now $h = (b,c)$ and the $\sigma_z$ is the third pauli matrix acting on flavour. These terms cause states, for example $| B \rangle$ to mix with states $|n'\rangle$, each being of the order of at least one of the following: $1/2m_b$,$1/2m_c$, and $(1/2m_b - 1/2m_c)$. It can be shown (\cite{Lebed:1991sq}) that the leading order symmetry breaking terms can only mix pseudoscalar and vector mesons with other irreps of the heavy quark symmetries. Hence, for example the example of the $B \to D$ transition, where
\begin{align}
	\langle D | \bar{c} b | B \rangle, \langle D | \bar{c}\gamma_{\mu} b | B \rangle &= 1 + \order{\left(1\over 2m_b\right)^2} 
	+ \order{\left(1\over 2m_c\right)^2} \\ &+ \order{\left( {1\over 2m_b} - {1\over 2m_c} \right)^2}.
\end{align}
this property extends straightforwardly to the form factors associated with these decays.

\subsubsection{Decay Constants}

{\color{red}how does $f_{B_c}$ vary with $m_b$?}

\subsection{NRQCD}

\section{Form Factors}

We now turn to some more technical details about semileptonic decays.
\\ \\
We will refer to the 4-momenta and mass of the initial and final states as $p_{1,2},M_{1,2}$ respectively, and define the 4-momentum taken away by the $W^{\pm}$ boson as $q \equiv p_2 - p_1$. We work in the rest frame of the initial meson, in which
\begin{align}
	q^2 = M_1^2 + M_2^2 - 2M_1 E_2.
\end{align}
There is a physically allowed range of values for an on-shell $q^2$. The minimum is when all of the 3-momentum of the initial state is taken by the final meson, $q_{\text{min}} ^2 = 0$. $q^2$ is maximised when all of the 3-momentum is taken by the boson, $\underline{p}_2^2 = 0 \rightarrow E_2 = \sqrt{M_2^2 + \underline{p}_2^2 } = M_2 \rightarrow q_{\text{max}}^2 = M_1^2 - M_2^2 - 2M_1M_2 = ( M_2 - M_1 )^2$. So
\begin{align}
	0 \leq q^2 \leq ( M_2 - M_1 )^2
\end{align}
This also creates an allowed range for the final meson 3-momentum:
\begin{align}
	0 < \underline{p}_{2}^2 < \left( M_1^2 + M_2^2 \over 2M_1 \right)^2 - M_2^2
\end{align}
\\ \\
Hadronic matrix elements like $H_{\mu}$ from sec. \ref{sec:cp} can be parameterised in terms of form factors. The current operator between between the states is a conserved current, so then the matrix element must be proportional to only conserved quantities, namely, elements of the stress-energy tensor. Lorentz invariance requires indices on either side of such a relation match, so a matrix element with a single Lorentz index can only be proportional to 4-momenta.
\\ \\
Let us consider the case where the two states in the matrix element are pseudoscalar mesons (as in $B_s\to D_s l\nu$), with an insertion of a left-handed current (i.e. the coupling to the $W^{\pm}$ in \eqref{eq:weakL}), $J_{\mu}^{ij} = \bar{u}_i \gamma_{\mu} {1\over2}(1-\gamma_5) d_j$. This can be written as $J^{ij}_{\mu} = V^{ij}_{\mu} - A^{ij}_{\mu}$, these are the vector and axial vector currents. 
\\ \\
The axial vector evaluated between two pseudoscalar states must vanish because the combination is not parity invariant thus does not contribute in pure QCD, leaving just the vector current. The most common parameterisation is:
\begin{align}
\label{eq:formfactors}
\langle P_2 (p_2) | V^{\mu} | P_1 (p_1) \rangle  &= f_{+} (q^2) \left[ p_1^{\mu} + p_2^{\mu} - {M_1^2 - M_2^2 \over q^2} q^{\mu} \right] + f_0(q^2) {M_1^2 - M_2^2 \over q^2 } q^{\mu}
\end{align}
Where $|P_i(p_i)\rangle$ is a pseudoscalar meson state with momentum $p_i$.

\subsection{Analyticity}
\subsection{z-Expansion}


\section{Renormalization of Currents}


The currents $V_{\mu}$ and $A_{\mu}$ are conserved in the Chiral limit (quark masses $\rightarrow 0$). This can be shown via Noether's theorem: consider a theory containing a number of fields $\{\varphi_i\}$, which transform under a group $G$ by
\begin{align}
	\varphi_i \rightarrow \varphi_i (x) - i\epsilon_a (x) F_{i,a}[\{\varphi\}]
\end{align}
It can be shown that there exists currents
\begin{align}
	J_a^{\mu} = -i{\partial \mathscr{L} \over \partial\partial_{\mu} \varphi_i} F_{i,a}
\end{align}
which obey
\begin{align}
	\label{eq:conservedJ1}
	J^{\mu}_a = {\partial \delta \mathscr{L} \over \partial \partial_{\mu} \epsilon_a } \\
	\partial_{\mu} J^{\mu}_a = {\partial \delta \mathscr{L} \over \delta \epsilon_a}	
	\label{eq:conservedJ2}
\end{align}
where $\delta\mathscr{L}$ is the result of an infinitesimal $G$ operation, i.e. $G$ : $\mathscr{L} \rightarrow \mathscr{L} + \delta\mathscr{L}$. From \eqref{eq:conservedJ2}, we see that if the theory is symmetric under the generator parameterised by $\epsilon_a$, then $J_a^{\mu}$ is a conserved current.
\\ \\
In the chiral limit, QCD with $N$ flavors has the accidental global symmetry ("chiral symmetry") $U(N)_L\times U(N)_R$:
\begin{align}
	q_{L/R} \rightarrow \text{exp}(-i\theta_a^{L/R} \lambda_a) q_{L/R}
	\label{eq:LRtransform}
\end{align} 
where $q$ is a vector in flavor space, and $\lambda_a$ are the generators of $U(N)$ in the fundemental representation and acts on flavor. Applying \eqref{eq:conservedJ1} and \eqref{eq:conservedJ2} to QCD we get the conserved currents:
\begin{align}
	J_{L/R}^{\mu,a} &= \bar{q}_{L/R} \gamma^{\mu} \lambda_a q_{L/R} \\
	\partial_{\mu} J_{L/R}^{\mu,a} &= 0
\end{align}
$J_{L/R}$ are often expressed instead in terms of {\it{vector}} and {\it{axial}} currents
\begin{align}
	V^{\mu,a} &= J^{\mu,a}_L + J^{\mu,a}_R = \bar{q} \gamma^{\mu} \lambda_a q \\
	A^{\mu,a} &= J^{\mu,a}_L - J^{\mu,a}_R = \bar{q} \gamma^{\mu} \gamma_5 \lambda_a q
\end{align}
which are conserved currents corresponding to the vector and axial symmetries $U(N)_V$ and $U(N)_A$, consisting of simultanious $L/R$ transformations above with constraints $\theta_a^{L} = \pm \theta_a^{R}$. Vector and axial currents between any two individal flavours, i.e.
\begin{align}
	\label{eq:individualflavs1}
	V^{\mu}_{ij} = \bar{q_i} \gamma^{\mu} q_j \\
	A^{\mu}_{ij} = \bar{q_i} \gamma^{\mu} \gamma_5 q_j
	\label{eq:individualflavs2}
\end{align}
are also conserved, since they can be constructed from linear combinations of $V^{\mu,a}$ and $A^{\mu,a}$.
\\ \\
$U(N)$ can be broken up into $SU(N)\times U(1)$, where $U(1)$ is a singlet transformation, i.e., of the form of \eqref{eq:LRtransform} with $\lambda = 1$. When QCD is quantized, it develops an anomaly which breaks the singlet axial symmetry. This reduces the symmetry group to $SU(N)_V\times SU(N)_A\times U(1)_V$, and prevents the corresponding axial singlet current $A^{\mu,0}$ from being conserved.
\\ \\
In the case of non-zero quark mass, the chiral symmetry is broken. This leads to \eqref{eq:conservedJ2} for the vector and axial currents evaluating instead as
\begin{align}
	\partial_{\mu} V^{\mu,a} &= i\bar{q} [ M,  \lambda_a ] q \\
	\partial_{\mu} A^{\mu,a} &= i\bar{q} \{ M, \lambda_a \} q 
\end{align}
where $M$ is the quark mass matrix in flavor space. These are the parially conserved axial and vector current identities (PCAC and PCVC). By taking a linear combination of these equations, one can derive PCAC and PCVC for individual flavors:
\begin{align}
	\partial_{\mu} V_{ij}^{\mu} &= i(m_i - m_j) \bar{q_i} q_j \equiv i(m_i - m_j) S_{ab} \\
	\partial_{\mu} A_{ij}^{\mu} &= i(m_i - m_j) \bar{q_i} \gamma_t q_j \equiv i(m_i - m_j) P_{ij}
\end{align}
We refer to $S$ as the scalar current and $P$ as the pseudoscalar. These identities translates straightforwardly to expectation values:
\begin{align}
	\label{eq:PCVC}
	(p_1 - p_2)_{\mu} \langle \psi(p_2) | V^{\mu}_{ij} | \phi(p_1) \rangle &= (m_i - m_j) \langle \psi(p_2) | S_{ij} | \phi(p_1) \rangle \\
	(p_1 - p_2)_{\mu} \langle \psi(p_2) | A^{\mu}_{ij} | \phi(p_1) \rangle &= (m_i - m_j) \langle \psi(p_2) | P_{ij} | \phi(p_1) \rangle
\end{align}
where $| \psi(p) \rangle$ and $| \phi(p) \rangle$ are arbitrary states containing momentum $p$.
\\ \\
These are powerful tools for connecting expectation values of different currents. For example, by combining \eqref{eq:PCVC} and \eqref{eq:formfactors}, we see that:
\begin{align}
	\langle P_2(p_2) | S_{ij} | P_1(p_1) \rangle = {M_1^2 - M_2^2 \over m_i - m_j} f_0(q^2)
\end{align}
Hence we have an extra route to accessing $f_0$. In a lattice simulation which computes expectation values of $V^{\mu}$, one could constrain $f_0$ by also computing the scalar current on the lattice, then use the rest of the information in $V^{\mu}$ to constrain $f_+$.