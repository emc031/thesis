\chapter{Conclusions}

In this work, we produced the first published results of applying the heavy-HISQ approach to semileptonic form factors. We found a new determination of $h_{A_1}^s(1)$, twice as accurate as the previous result and containing considerably fewer assumptions -
\begin{align}
  \mathcal{F}^{B_s\to D_s^*}(1) = h^s_{A_1}(1) = 0.9020(96)_{\text{stat}}(90)_{\text{sys}}\,.
\end{align}
Future experimental data may be combined with this result to produce a new determination of $|V_{cb}|$.

We found $B_s\to D_sl\nu$ form factors using heavy-HISQ, also improving on the precision in comparison to previous lattice results. For the first time, we were able to obtain lattice data spanning the entire $q^2$ range, due to the properties of the heavy-HISQ method. From these form factors, we found
\begin{align}
  R_{D_s}|_{\text{SM}} = 0.2985(43)_{\text{stat}}(27)_{\text{sys}}.
\end{align}
which can be combined with future experimental data to provide a new test of the Standard Model, namely a new probe into the possibility of lepton flavour violation.

Besides the successes from the heavy-HISQ approach, I have discovered some problems with using NRQCD-HISQ currents to compute semileptonic $b\to c$ transitions on the lattice. Namely, it was discovered that the expansion of NRQCD-HISQ currents do not converge very fast. $\order{1/m_b}$ and possibly $\order{1/m_b^2}$ terms are important for the dispersion relation of heavy-light mesons. So-called negligible pieces of the spacial vector current have large magnitudes:
\begin{align}
  V_k^{(2)}\sim V_k^{(4)} \sim 0.35\times V_k^{(0)}\,.
\end{align}
I attempted some approaches to non-perturbatively renormalizing the NRQCD-HISQ currents in order to account for these issues, with limited success. 

Heavy-HISQ calculations are more computationally costly than their equivalent calculations using NRQCD for the $b$. Taking into account the need for finer lattices and multiple data with different $m_h$ values, heavy-HISQ costs something of the order of 20 times more than NRQCD. However, the cost in human time of NRQCD, via perturbative matching calculations and larger and less stable correlator fits, is clearly larger than for heavy-HISQ. Also, heavy-HISQ does not contain assumptions of negligible subleading terms in the relativistic expansion or validity of perturbation theory via the matching.

The dominant uncertainty in our heavy-HISQ results are statistical. Future calculations with this approach must increase the statistics of lattice data to improve on the results presented here. This means gaining data on more gauge configurations, and with more choices of source timeslice $t_0$. This will simply require more computational resources to achieve.

These two resuls from heavy-HISQ are in strong agreement with all other recent lattice determinations of these form factors and the analogous $B\to D^*\ell \nu$ and $B\to D\ell\nu$ form factors. Other lattice determinations use very different methodologies to the work presented here, compare for example our $h_{A_1}^s(1)$ calculation to the Fermilab/MILC calculation of $h_{A_1}(1)$. That calculation used different gauge ensembles ($N_f=2+1$ MILC), a different action for the $b$ and $c$ quarks (Fermilab action), different approach to analyzing correlation functions (double-ratio approach), different normalization of currents (perturbative normalization), and different continuum and light mass extrapolation. The combination of consistent results from independent studies makes the overall contribution of lattice QCD to $b\to c$ form factors extremely robust.

Further contributions of form factors from lattice QCD are necessary in $b\to c$ transitions. The current precision on $|V_{cb}|$ is limited in roughly equal part by theoretical and experimental errors, so more precision on $b\to c$ form factors is needed to more precisely determine $|V_{cb}|$, and understand the source of the tension in its exclusive/inclusive determinations. The SM predictions of $R(D_{(s)}^{(*)})$ are currently much more precise than the experimental measurements, however more independent SM calculations of these ratios are necessary to ensure no errors are being underestimated.

%Going into the future, where computational cost becomes less of an issue, it is clear which of these is the smart choice. Let no one ever have to scratch their head over such a choice again.

