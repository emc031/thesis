\chapter{Lattice Calculations}
\label{chap:latticecalculations}

\section{Correlation Functions from Lattice Simulations}

A typical quantity that is computed on the lattice is a meson correlation function, i.e. when $\mathcal{O} = \Phi(x)\Phi^{\dagger}(y)$ and $\Phi$ is a meson creation operator. This is a good working example for showing the steps in a lattice calculation. 
\\ \\
A creation operator for a meson in this context can be any operator containing the same quantum numbers as the meson one is trying to create. For example, the neutral $B$ meson is a pseudoscalar charged with a $d$ and $\bar{b}$ quark, so a suitable operator is $\Phi(x) = \bar{b}(x)\gamma_5 d(x)$. The path integral can then be written as 
\begin{align}
	C(x,y) = \int \mathcal{D}\psi \mathcal{D}\bar{\psi} \mathcal{D}U \left(\text{ } \bar{b}(x)\gamma_5 d(x) \bar{d}(y)\gamma_5 b(y) \text{ }\right) e^{-S_G[U]-\sum_{w,z} \bar{\psi}(w)M(w,z)[U]\psi(z)}
\end{align}
where we have now broken the action up into a gauge part $S_G[U]$, and a fermion part. $M(x,y)[U]$ is the Dirac operator, and can be seen as a matrix in lattice site, flavor, color and spin. $\psi$ is a vector of quark flavours.
\\ \\
The integral over fermions can be preformed analytically, since the fermion fields are Grassman valued. In our example, the result is \cite{Peskin:1995ev},
\begin{align}
	C(x,y) = \int \mathcal{D}U \text{ Tr}\left[\text{ } M^{-1}_b(y,x)[U] \gamma_5 M^{-1}_d(x,y)[U] \gamma_5 \text{ }\right] e^{-S_G[U]} \text{det}(M[U])
\end{align}
The quantinties $M^{-1}_f(x,y)[U]$ are propagators of a quark of flavour $f$ on a fixed gauge background $U$. The integration over gauge fields is generally carried out by an importance sampling method. A finite \textit{ensemble} of gauge configurations $\{U_i\}$ is generated by a Monte Carlo Markov chain (MCMC), where the probability of a gauge configuration $U_j$ being added to the ensemble is proportional to 
\begin{align}
	e^{-S_G[U_j]}\text{det}(M[U])
	\label{eq:MCweight}
\end{align}
See \cite{DeGrand:2006zz} ch. 7 for examples of such algorithms. In the case of our work, we use ensembles generated by the MILC collaboration \cite{Bazavov:2012xda}.
\\ \\
Once the ensemble is created, the path integral can be approximated by simply
\begin{align}
  C(x,y) \simeq {1\over N} \sum_i \text{ Tr}\left[\text{ } M^{-1}_b(y,x)[U_i] \gamma_5 M^{-1}_d(x,y)[U_i] \gamma_5 \text{ }\right]
  \label{eq:av_gauge}
\end{align}
where $N$ is the size of the ensemble. The calculation of the correlation function then is split into 3 steps:
\begin{enumerate}
	\item
	Generate an ensemble of Gauge configurations $\{ U_i \}$ by MCMC.
	\item
	Compute quark propagators $M^{-1}_f(x,y)[U]$ on each Gauge configuration. This requires inverting the matrix $M$ each time, this is typically done by conjugate gradient method.
	\item
	Construct trace as in \eqref{eq:av_gauge}, and average over the ensemble.
\end{enumerate}
We now turn to the issue of choosing lattice actions.

\subsection{Path Integral}
\subsection{Dirac Operator Inversion}
\subsection{Random Wall Sources}

%% preliminaries

The full set of spin-mixing matrices can be labelled according to
\begin{align}
	\gamma_n = \prod_{\mu} \left( \gamma_{\mu} \right)^{n_{\mu}} \quad n_{\mu} = \mathbb{Z}_2
\end{align}
There are 16 such matrices representing corners of the hypercube. As $\gamma_{\mu}^2=1$, one can also use a general site vector $x_{\mu}$ to label the matrix, then $\gamma_x = \gamma_n$ where $n_{\mu} = x_{\mu} \mod 2$. One can show that for any $x$; $\gamma_x^{\dagger} \gamma_x = 1$.
\\ \\
Naive quarks $\psi(x)$ can be transformed into staggered quarks $\chi(x)$ via $\psi(x) = \gamma_x \chi(x)$. Then, Naive quark propagators (inverse Dirac operators) become
\begin{align}
	G_{\psi}(x,y) = \gamma_x\gamma^{\dagger}_y G_{\psi}(x,y)
\end{align}
By conjugating both sides and using $\gamma_5$-hermiticity $G^{\dagger}_{\psi}(y,x) = \gamma_5 G_{\psi}(y,x)\gamma_5$ it can be shown that
\begin{align}
	\label{eq:Gconj}
	G_{\psi}(x,y) = \phi_5(x)\phi_5(y) G^{\dagger}_{\psi}(y,x)
\end{align}
where $\phi_5(x) = (-1)^{\sum_{\mu} x_{\mu}}$.
\subsubsection{2pt correlation functions}
We will break down the correlation function to see what quantities must be computed by the simulation. Consider the generic 2-point correlator:
\begin{align}
	C(x,y) &= \langle \Phi^{\dagger}_X(x) \Phi_Y(y) \rangle_{\psi,U} \quad , \quad \Phi_X(x) = {1\over 4}\bar{\psi}_a(x) \gamma_X \psi_b(x) \\
	&= {1\over 16}\langle tr_{c,s} \gamma_X G_{a,\psi}(x,y) \gamma_Y G_{b,\psi}(y,x) \rangle_U \\
	&= {1\over 16}tr_s \left( \gamma^{\dagger}_x \gamma_X \gamma_x \gamma_y^{\dagger} \gamma_Y \gamma_y \right)
	\langle tr_c \left( G_{a,\chi}(x,y) G_{b,\chi}(y,x) \right) \rangle_U
\end{align}
$tr_s$ is a trace over spin and $tr_c$ is over colour. To deal with the spin trace, define the family of phases $\{\phi_X(x)\}$ according to
\begin{align}
	\gamma^{\dagger}_x\gamma_X\gamma_x = \phi_X(x) \gamma_X
\end{align}
for example, if $X=5$, then $\gamma^{\dagger}_x\gamma_5\gamma_x = (-1)^{\sum_{\mu}x_{\mu}} \gamma^{\dagger}_x\gamma_x \gamma_5 = \phi_5(x) \gamma_5$. The map from $X$ to $\phi_X$ is structure preserving, i.e. if $\gamma_X=\gamma_A\gamma_B$, then $\phi_X(x)=\phi_A(x)\phi_B(x)$. The spin trace becomes $\phi_X(x)\phi_Y(y) tr_s\left( \gamma_X \gamma_Y \right)$. The will vanish unless $Y=X$, as one would expect physically for the correlation function. We end up with
\begin{align}
	C(x,y) = {1\over 4} \phi_X(x)\phi_X(y) \langle tr_c G_{a,\chi}(x,y) G_{b,\chi}(y,x) \rangle_U
\end{align}
It is useful in the simulation to replace $G_{b,\chi}(y,x)$ with it's conjugate via \eqref{eq:Gconj}, resulting in
\begin{align}
	C(x,y) = {1\over 4} \phi_{5X}(x)\phi_{5X}(y) \langle tr_c G_{a,\chi}(x,y) G^{\dagger}_{b,\chi}(y,x) \rangle_U
\end{align}
where $\phi_{5X}(x) = \phi_5(x)\phi_X(x)$. To obtain the correlation function of a meson in an eigenstate with momentum ${\underline{p}}$, the above must be replaced with
\begin{align}
	C_{\underline{p}}(t_0,t) &= {1\over L^3} \sum_{\underline{x},\underline{y}} e^{i\underline{p}\cdot(\underline{x}-\underline{y})}
	C(\underline{x},t_0;\underline{y},t) \\
	&= {1\over 4 L^3} \sum_{\underline{x},\underline{y}} e^{i\underline{p}\cdot(\underline{x}-\underline{y})} \phi_{5X}(x)\phi_{5X}(y) \langle tr_c G_{a,\chi}(x,y)G^{\dagger}_{b,\chi}(y,x)\rangle_U,
\end{align}
where it is understood that $x_0=t_0$ and $y_0=t$. In order to evaluate this function, the simulation must perform inversions to create $G_{a/b,\chi}(x,y)$ for each $x$ and $y$, so $2\cdot$Vol$^2$ operations. This is prohibitively expensive. The number of inversions can be reduced using {\it{random wall sources}}. Define
\begin{align}
	P^{t_0}_{a,\underline{p},X}(y) \equiv {1\over \sqrt{L^3}} \sum_{\underline{x}} e^{i\underline{p}\cdot (\underline{x}-\underline{y})} \phi_{5X}(\underline{x},t_0) \xi(\underline{x}) G_{a,\chi}(\underline{x},t_0;y),
\end{align} 
where $\xi(\underline{x})$ is a random field of colour vectors, varying with $U$. This has the property
\begin{align}
	\langle f(\underline{x},\underline{x}') \xi^*(\underline{x}')\xi(\underline{x})\rangle_U = \delta_{\underline{x},\underline{y}} \langle f(\underline{x},\underline{y}) \rangle_U.
\end{align}
Using this property it can be shown that the correlator can be build instead according to
\begin{align}
	C(\underline{x},t_0;\underline{y},t) \simeq {1\over 4} \sum_{\underline{y}} \phi_{5X}(y) \langle tr_c P^{t_0}_{a,\underline{p},X}(\underline{y},t)P^{t_0,\dagger}_{b,\underline{0},1}(\underline{y},t)\rangle_U
	\label{eq:tietogether}
\end{align}
Now all the simulation has to do is compute $P^{t_0}_{a/b}(y)$ for general $y$, so $2\cdot$Vol operations, a reduction by a factor of Vol. 
\\ \\
In the MILC code, "sources" are first created (the fields $\phi_{5X}(\underline{x},t_0) \xi(\underline{x})$), then the objects $P^{t_0}(y)$ (referred to as "propagators") are generated from them. Any extra factors dependant on $y$ (this is useful for "smeared" propagators, see {\color{red}?}) can be multiplied in. The resulting object $f(y)\cdot P^{t_0}(y)$ is referred to as a "quark". Finally, two of these quarks can be "tied together" according to \eqref{eq:tietogether}, to produce correlation functions. The sources are chosen to be on some single timeslice $t_0$, resulting in a value for $C(t_0,t)$ at each $t$. 

%% 3-point correlators

The above discussion can be generalized to $3-$(or $N-$)point correlation functions using {\it{extended sources}}. Consider a 3-pt correlator encoding the form-factors of a semileptonic decay from meson $X$ to meson $Z$, via a current $J$. We start by evaluating
\begin{align}
	C(x,y,z) = \langle \Phi^{\dagger}_X(x) J(y) \Phi_Z(z) \rangle_{\psi,U} \quad , \quad \Phi_X(x) &= {1\over 4} \bar{\psi}_b(x) \gamma_X \psi_s(x) \\
	\nonumber
	J(y) &= \bar{\psi}_b(y)\gamma_J\psi_a(y) \\
	\nonumber
	\Phi_Z(z) &= {1\over 4}\bar{\psi}_a(z)\gamma_Z \psi_s(z)
\end{align}
in the same way as before:
\begin{align}
	C(x,y,z) &= {1\over 16} tr_s\left( \gamma^{\dagger}_x \gamma_X \gamma_x \gamma^{\dagger}_y \gamma_J \gamma_y \gamma^{\dagger}_z \gamma_Z \gamma_z \right) \langle tr_c G_{b,\chi}(x,y) G_{a,\chi}(y,z) G_{s,\chi}(z,x)\rangle_U \\
	&= {1\over 4} \phi_{5X}(x) \phi_J(y) \phi_{5Z}(z) \langle tr_c G_{b,\chi}(x,y) G_{a,\chi}(y,z) G^{\dagger}_{s,\chi}(z,x) \rangle_U
\end{align}
We have assumed that $tr_s \gamma_X\gamma_J\gamma_Z = 4$, requiring that each gamma matrix in this combination has a partner and therefore cancels. In any other situation the trace would vanish. For example, if the current is a temporal vector $J=0$, and the two mesons represent pseudoscalars, one of the meson operators must have a $\gamma_0$, i.e. one could choose $\gamma_X=\gamma_0\gamma_5, \gamma_Z=\gamma_5$. {\color{red}why is it ok to have a non-goldstone for $X$?}
\\ \\
Putting $X$ into an eigenstate of zero momentum, and $Y$ into an eigenstate of momentum $\underline{p}$, we get
\begin{align}
	C_{\underline{p}}(t_0,t,T) = {1\over 4 L^3} \sum_{\underline{x},\underline{y},\underline{z}} e^{i\underline{p}\cdot(\underline{y}-\underline{z})} \phi_{5X}(x) \phi_J(y) \phi_{5Z}(z) \langle tr_c G_{b,\chi}(\underline{x},t_0;\underline{y},t) G_{a,\chi}(\underline{y},t,\underline{z},T) G^{\dagger}_{s,\chi}(\underline{z},T;\underline{x},t_0) \rangle_U
	\label{eq:3ptfullexpr}
\end{align}
This can be built by first creating propagators for the $b$ and $s$ quarks: $P^{t_0}_{b,\underline{0},X}(y)$,$P^{t_0}_{s,\underline{0},1}(z)$. Then, build the $a$ propagator using an extended source, i.e.:
\begin{align}
	P^T_{a,\underline{p},ext}(y) = \sum_{\underline{z}} P^{t_0}_{s,\underline{0},1}(\underline{z},T) \phi_{5Z}(\underline{z},T) G_{a,\chi}(\underline{z},T;y)
\end{align}
Then, by plugging $P^{t_0}_{b,\underline{0},X}(y)$ and $P^T_{a,\underline{p},ext}(y)$ into the MILC tie-together defined by \eqref{eq:tietogether},one ends up evaluating \eqref{eq:3ptfullexpr}.


\section{Analysis of Correlation Functions}
\subsection{Non-Linear Regression}


Once a correlation function like the in \ref{sec:correlators} has been computed, we can extract physics from it, namely the mass and decay constant of the meson we are studying. In practice the meson creation operators defined above are fourier transformed
\begin{align}
	\Phi(\underline{k},t) = \sum_{\underline{x}} e^{-i\underline{k}\cdot\underline{x}} \Phi(\underline{x},t)
\end{align}
which serves to change \eqref{eq:av_gauge} into 
\begin{align}
	C_{\underline{k}}(t) = {1\over N} \sum_i \sum_{\underline{x},\underline{y}} e^{-i\underline{k}\cdot(\underline{x}-\underline{y})} \text{ Tr}\left[\text{ } M^{-1}_b(\underline{y},t;\underline{x},0)[U_i] \gamma_5 M^{-1}_d(\underline{x},0;\underline{y},t)[U_i] \gamma_5 \text{ }\right]
	\label{eq:correlator}
\end{align}
\eqref{eq:correlator} is computed for many $t$ values with a lattice calculation following the principles detailed above. One performs a least-squares fit of 
this to a theoretically motivated function of $t$. To derive such a function,
first construct a complete set of momentum $\underline{k}$ states with quantum numbers matching the meson:
\begin{align}
 1 = \sum_{n=0} {1\over 2E^r_n} | \lambda_n \rangle \langle \lambda_n |.
\end{align}
Where $E^r_n = \sqrt{ M_n^2 + \underline{k}^2}$ are the relativistic energies of each state. Inserting this into the correlation function, and moving from the Heisenberg to Schroedinger picture;
\begin{align}
	C_{\underline{k}}(t) &= \sum_{n=0} {1\over 2E^r_n} \langle 0 | e^{Ht} \Phi(\underline{k},0) e^{-Ht} | \lambda_n \rangle \langle \lambda_n | \Phi^{\dagger}(\underline{k},0) | 0 \rangle
	\nonumber
	\\ &= \sum_{n=0}  \left( {\langle 0 | \Phi(\underline{k},0) | \lambda_n \rangle \over \sqrt{2E^r_n}} \right) \left( {\langle \lambda_n | \Phi^{\dagger}(\underline{k},0) | 0 \rangle \over \sqrt{2E^r_n} } \right) e^{-E^l_n t}
	\nonumber
	\\ & \equiv \sum_{n=0} |a_n|^2 e^{-E^l_n t}.
	\label{eq:multiexponential}
\end{align}
The fit results in a determination of the parameters $a_n$, and $E^l_n$. Since the lowest energies dominate the function at late times, one can afford to truncate the sum over $n$ to some tractable range, in our case $n\in[1,6]$. We interpret $|\lambda_0\rangle$ to be the ground state of the meson we are studying. The exponential decays mean the fit function is dominated by the ground state at large $t$, and subsequent excited states become less important as $E^l_n$ increases. Hence by only including $C_{\underline{k}}(t)$ at suitably large $t$ values, we can affort to truncate the sum in $n$. In our fits we chose $n=6$.
\\ \\
We maintain a distinction between $E^l$ and $E^r$, since for example in simulations involving NRQCD quarks these differ. If this is not an issue, as is the case with HISQ, one can compute the correlation function at zero momentum $C_{\underline{0}}(t)$, then fit it to find the parameter $E^l_0$, which will equal the meson's mass $M$. Noting the definition of a meson decay constant $f$: $\langle 0 | J_0 | \text{Meson}(\underline{k}=0) \rangle = M f$, where $J_0$ is a temporal current with the same quantum numbers as the meson, we can see that the fit parameters $a_n$ at zero momentum are related to the meson's decay constant via
\begin{align}
	f = \sqrt{2\over M} a_0
\end{align}
Hence the fit can also be used to extract decay constants.
\\ \\
The above discussion can be straightforwardly generalized to 3-point correlation functions, from which we are able to extract quantities like the hadronic transition amplitudes $H_{\mu} = \langle M_{q_1\bar{q_3}} | J_{\mu}^{q_1\bar{q}_2} | M_{q_2\bar{q}_2} \rangle$ from sec. \ref{sec:cp}. Specifically the quantity we require in order to deduce the $B_{s}\to D_{s} l\nu$ form factors is $\langle D_{s} | V_{\mu} | B_{s} \rangle$, where $V_{\mu}=\bar{c}\gamma_{\mu} b$. 
\\ \\
The generalization of the above for 3pt functions is summarized here:
\begin{align}
	C_3(t,T) &= \int \mathcal{D}\psi \mathcal{D}\bar{\psi} \mathcal{D}U \left(\text{ } \Phi_{D_s}(\underline{0},0) V_{\mu}(-\underline{p},t) \Phi_{B_s}^{\dagger}(\underline{p},T) \text{ }\right) e^{-S[\psi,\bar{\psi},U]} \\
	&\simeq {1\over N} \sum_i \sum_{\underline{x},\underline{y},\underline{z}} e^{-i\underline{p}\cdot (\underline{y}-\underline{z})} \text{ Tr}\left[\text{ } M^{-1}_b(\underline{x},0;\underline{y},t)[U_i] \gamma_{\mu} M^{-1}_c(\underline{y},t;\underline{z},T)[U_i]\gamma_5\gamma_5 M_s^{-1\dagger}(\underline{z},T;\underline{x},0)[U_i]  \text{ }\right] \\	
	&= \sum_{n,m}  \left( {\langle 0 | \Phi_{D_s} | \lambda_n \rangle \over \sqrt{2E^r_n}} \right) \left({\langle \lambda_n | V_{\mu} | \lambda_m \rangle \over 2\sqrt{ E^r_n E^r_m } }\right) \left( {\langle \lambda_m | \Phi_{B_s}^{\dagger} | 0 \rangle \over \sqrt{2E^r_n} } \right) e^{-E^l_m (T-t)} e^{-E^l_n t} \\
	\nonumber
	& \equiv \sum_{n,m} a_{D_s,n} V_{nm} a^*_{B_s,m} e^{-E^l_m (T-t)} e^{-E^l_n t}.
	\label{eq:3ptfitfunction}
\end{align}
$C(t,T)$ is computed at different values of $t$ and $T$, then a least-squares fit is performed to the fit function \eqref{eq:3ptfitfunction}. $a_n$ will vanish for states $|\lambda_n\rangle$ which have different quantum numbers to $\Phi_{B_s}$, and similarly for $b_m$ and $\Phi_{D_s}$. Non-zero $a_n$'s will match the analagous parameters extracted from fitting a 2pt function $\langle \Phi_{B_s}^{\dagger} \Phi_{B_s} \rangle$, similarly for $b_n$'s and $\Phi_{D_s}$. This carries on to the energies, $\{E^l_n\}$ is the spectrum for the $D_s$ meson, and $\{E^l_m\}$ is the spectrum for the $B_s$. Therefore, we compute and fit the appropriate 2pt functions to deduce the parameters $\{a_n\}$,$\{b_m\}$,$\{E^l_n\}$. Then, fitting $C_3(t,T)$ results in an accurate determination of the remaining free parameters, $V_{nm}$. This set contains the quantity we need, recoginising that:
\begin{align}
	V_{00} = {\langle D_s | V_{\mu} | B_s \rangle \over 2\sqrt{E^{B_s} E^{D_s} } }
\end{align}

%% now with oscillations


2-point correlation functions are then fitted as in sec. \ref{sec:fitting}. The fit function we use is modified a little from \eqref{eq:multiexponential}, we use:
\begin{align}
	\nonumber
	C^{\alpha\beta}(t) &= \sum_{n} a^{\alpha*}_n  a^{\beta}_n ( e^{-E^l_n t} - se^{-E^l_n(T-t)} )\\
	& + \sum_{n} a^{'\alpha*}_n a^{'\beta}_n (-1)^t ( e^{-E^{'l}_n t} - se^{-E^{'l}_n(T-t)} )
	\label{eq:2ptcorrelator_real}
\end{align}
Firstly, the parameters $\{a_n\}$ must vary between source and sink to account for the different operators. Secondly, the periodicity of the lattice in the time direction means an extra exponential term is required, but not in the case of the $B_s$ since NRQCD quarks do not experience the periodicity of the lattice. Hence $s$ is set to 0 for the $B_s$ correlator and 1 for the $D_s$. $T$ is the time extent of the lattice. The second term is to account for the so-called "oscillating state", which is in fact the $\zeta=(1,0,0,0)$ doubler fermion appearing due to the doubling in the HISQ action (see sec. \ref{eq:doubling}). No other doublers contribute, since $\Phi_{\underline{k}}$ has a 3-momentum fixed at $\underline{k}$, which we always take to be small relative to $\pi/a$, hence does not couple to the states at $k\sim(0,\pi/a,0,0)$, $k\sim(0,0,\pi/a,0)$ etc. However, $\Phi_{\underline{k}}$ can couple to arbitrarily high energy states, so the $k\sim(\pi/a,0,0,0)$ doubler contributes. The second term in \eqref{eq:2ptcorrelator_real} is justified by performing the doubling operation $\mathcal{B}_0$ defined in \eqref{eq:multiexponential}, the quark fields in $\Phi_{\underline{k}}$ which obey the HISQ action. See appendix G of \cite{Follana:2006rc} for details.
\\ \\
We use the \textit{CorrFitter} package \cite{CorrFitter} for performing Baysian least-squares fitting to the correlation functions. The package employs the trust region method of least-squares fitting. The fits require priors for each of the fit parameters. The "amplitude" parameters $a_{n,B_s/D_s}^{\alpha}$ are given priors of $0.1(1.0)$, thus inserting only the assumption that they are of $\mathcal{O}(1)$. The ground state energies are given priors motivated by the meson masses, and excited state energies are given loose, evenly spaced priors with ~600MeV between each level.
\\ \\
2-point correlation functions for $B_s$ and $D_s$ are fit to \eqref{eq:2ptcorrelator_real}, resulting in $a^{\alpha}_{n,B_s/D_s}$,$E^l_{n,B_s/D_s}$. Since the HISQ action is fully relativitstic, $E^l_{0,D_s}$ at $\underline{k}=0$ can be interpreted as the $D_s$ meson mass. The same cannot be done for the $B_s$. The decay constants for $B_s$ and $D_s$ can be deduced from $a^0_{0,B_s}$ and $a^0_{0,D_s}$, since $\Phi^0_{B_s,D_s}$ are also temporal axial currents. This is a good avenue for consistency checks, we compared $M_{D,s}$,$f_{D_s}$ and $f_{B_s}$ to those computed in \cite{Colquhoun:2015oha},\cite{Monahan:2017uby} amongst others, and found them to be consistent (modulo small shifts we can reasonably expect due to differing choices of bare quark masses).
\\ \\
We now discuss fitting the 3-point correlation functions. The same considerations as those that went into \eqref{eq:2ptcorrelator_real} lead us to our 3-point fit function:
\begin{align}
	\nonumber
	C^{\alpha\beta}_3(t,T) =& \sum_{k,m} \big( a_{k,D_s}^{\alpha} V^{nn}_{km} a_{m,B_s}^{*\beta} e^{-E^l_m t} e^{-E^l_k (T-t)} \\
	\nonumber
	& + a_{k,D_s}^{\alpha} V^{no}_{km} a_{m,B_s}^{'*\beta} e^{-E'^l_m t} e^{-E^l_k (T-t)} \\
	\nonumber
	& + a_{k,D_s}^{'\alpha} V^{on}_{km} a_{m,B_s}^{*\beta} e^{-E^l_m t} e^{-E'^l_k (T-t)} \\
	& + a_{k,D_s}^{'\alpha} V^{oo}_{km} a_{m,B_s}^{'*\beta} e^{-E'^l_m t} e^{-E'^l_k (T-t)} \big)
	\label{eq:3ptcorrelator_real}
\end{align}
The 2-point and 3-point correlators are fit simultaniously, according to fit functions \eqref{eq:2ptcorrelator_real} and \eqref{eq:3ptcorrelator_real}. The parameters involved in the 2pt fits are mostly fixed by the data in the 2pt correlation functions, so the fit can use most of the data in the 3pt correlation functions to determine the transition amplitudes $V^{ab}_{km}$. This is carried out for each $B_s$ and $D_s$ smearing, each direction $\mu$ and each current correction $i$ of the vector current $V^{(i)}_{\mu}$.
\\ \\
In this large 2pt/3pt fit, there is a huge $\chi^2$ manifold with many local minema, and it is crucial to impose strong priors in order to ensure the fit finds the true minemum. Priors for ground state 2-point amplitudes and energies $a_{0,B_s/D_s}^{\alpha}$,$E^l_{0,B_s/D_s}$ are taken to be the results from individual fits of the 2-point functions, with the errors expanded by a factor of 2. The excited state amplitudes and energies are given the same priors as in the 2-point fits. The transition amplitueds $V^{ab}_{km}$ are given the prior 0.1(1.0), assuming it to be $\mathcal{O}(1)$.
\\ \\
Finally, we end up with the sought-after parameters $V_{00}^{nn}$ representing $V_{\mu}^{(i)}$, via the relation
\begin{align}
	\langle D_s | V^{(i)}_{\mu} | B_s \rangle =  2\sqrt{M^{B_s} E^{D_s}} V_{00}^{nn} |_{V=V_{\mu}^{(i)}\text{in simulation}}
\end{align}
We have asserted the ground state of $B_s$ to be it's mass, but not the $D_s$, as we give $D_s$ spacial momenta in the calculation (to be expanded on in the next section). Then, the full vector currents $\langle D_s | V^{(i)}_{\mu} | B_s \rangle$ can be build from a linear combination of these according to \ref{eq:currentcorrections}.


\subsection{Signal/Noise Ratio}


One of the main obstacles in our calculation is the \textit{signal degredation} of correlation functions computed on the lattice.
\\ \\
A random variable $x$ has mean and standard deviation
\begin{align}
	&\hat{x} = \langle x \rangle \\
	&\sigma^2 = {1\over N} ( \langle x^2 \rangle - \langle x \rangle^2 ),
\end{align}
where $N$ is the size of the sample. So the (square of) the signal/noise ratio is
\begin{align}
	{\hat{x}^2\over \sigma^2} = N \left( {\langle x^2 \rangle \over \langle x \rangle^2} - 1 \right) ^{-1}.
\end{align}
Consider 2 point correlators where $x = \Phi^{\dagger}(t) \Phi(0)$, where $\Phi$ is a meson operator with zero spacial momentum.
\\ \\
$\langle x^2 \rangle$ and $\langle x \rangle$ can be written as
\begin{align}
	\langle x \rangle &= \sum_k {1\over 2E_n} \langle 0 | \Phi^{\dagger}(t) | \lambda_n \rangle \langle \lambda_n | \Phi(0) | 0 \rangle e^{-E_n t} \simeq_{t\to\infty} e^{-E_0 t}\\
	\langle x^2 \rangle &= \sum_n {1\over 2E_n} \langle 0 | \Phi^{\dagger 2}(t) | \lambda_n \rangle \langle \lambda_n | \Phi^2(0) | 0 \rangle e^{-E_n t} \simeq_{t\to\infty} e^{-E'_0 t}
\end{align}
where we have assumed the ratio of matrix elements and energies are $\mathcal{O}(1)$. The two ground state energies $E_0$ and $E'_0$ need not be the same, since the lowest states for which 
$\langle \lambda_n | \Phi(0) | 0 \rangle \neq 0$ and $\langle \lambda_n | \Phi^2(0) | 0 \rangle \neq 0$ may differ. 
\\ \\
The operator $\Phi^2$ will contain two quark and two antiquark operators, connected by some matrices in spin space. $\Phi^2$ can create a combination of all possible 2 meson states where the mesons are made of the available quark species, and quantum numbers. For example, If $\Phi$ is a pion, $\Phi^2$ is a 2 pion state, and $E_0' = 2m_{\pi}$. If $\Phi$ is a $D_s$ meson, then $E_0' = m_{\tilde{\pi}} + m_{\eta_c}$ ($\tilde{\pi}$ is a pseudoscalar $s\bar{s}$ state).
\\ \\
Define $\mu_0 = E_0'/2$. Then
\begin{align}
	{\hat{x}^2\over \sigma^2} \simeq N \left( e^{-2(\mu_0 - m_{\Phi})t} - 1 \right) ^{-1}
\end{align}
In the case of pions, $\mu_0 = m_{\Phi}$, the ratio becomes simply $\sim N$. For mesons heavier than the pion, $\mu_0 < m_{\Phi}$, so at large times $e^{-2(\mu_0 - m_{\Phi})t} >> 1$, and upon taylor expanding the inverse of this phase we arrive at
\begin{align}
	{\hat{x}\over\sigma} \simeq \sqrt{N} e^{-(m_{\varphi} - \mu_0)t}
	\label{eq:signaltonoise}
\end{align}
From this we see there are 3 variables which effect the quality of the signal:
\begin{enumerate}
\item
The size of the sample $N$.
\item
At large $t$, the correlators undergo {\it{signal degredation}}, i..e, become dominated by noise.
\item
The degree of signal degradation is decided by $m_{\varphi} - \mu_0$. Heavier mesons will tend to experience more signal degredation.
\end{enumerate}
Relevant to our calculation is how giving mesons non-zero spacial momenta $\underline{p}$ can exaserbate this problem. In this case, $m_{\Phi}$ in \eqref{eq:signaltonoise} is replaced with $\sqrt{m_{\Phi}^2+\underline{p}^2}$. As $\underline{p}$ increases, the signal/noise ratio will degrade more and more and statistics will suffer. 
\\ \\
In the $B_s\to D_s l\nu$ calculation, to deduce form factors over the whole range of $q^2$ values, we need to simulate the process with the $D_s$ having a range of momenta $0 < |\underline{p}| < 2.32$GeV, as discussed in sec. \ref{sec:formfactors}. Correlation functions at the higher end of this range may be too noisy for any meanigful results to be extracted. We are investigating ways of taming this problem, see sec. \ref{sec:highmomenta}.


\section{Dealing with Heavy Quarks}
\subsection{Heavy HISQ}
\subsection{Lattice NRQCD}
\section{Renormalization of Currents}


Once one has computed a matrix element from a lattice calculation, it needs to be translated into a continuum regularization scheme. Suppose we have some bare operator $\mathcal{O}_0$, we expect this to be related to the renormalized operator in $\overline{MS}$ at scale $\mu$, $\mathcal{O}^{\overline{MS}}(\mu)$, via
\begin{align}
	\mathcal{O}^{\overline{MS}}(\mu) = Z^{\overline{MS}}(\mu) \mathcal{O}_0.
\end{align}
Similarly, in a lattice regularization,
\begin{align}
	\mathcal{O}^{\text{lat}}(1/a) = Z^{\text{lat}}(1/a) \mathcal{O}_0.
\end{align}
Hence we expect a multiplicative factor between the lattice matrix elements, and the continuum $\overline{MS}$ ones:
\begin{align}
	\langle \mathcal{O} \rangle^{\overline{MS}} = \mathcal{Z}(\mu,1/a) \langle \mathcal{O} \rangle^{\text{lat}}
\end{align}
where $\mathcal{Z}(\mu,1/a) = Z^{\overline{MS}}(\mu)/Z^{\text{lat}}(1/a)$. These "matching factors" $\mathcal{Z}$ can be deduced by equating observables calculated in both lattice QCD and continuum (appropriately regularized) QCD, producing equations which can be solved for $\mathcal{Z}$. The lattice side of the calculation can be done either through lattice perturbation theory ("perturbative matching"), or through a simulation ("non-perturbative matching").
\\ \\
It is a well-known result that conserved (or partially conserved) currents do not receive any renormalization in any scheme, i.e. $Z^{\text{any}}=1$ ("absolutely normalized").
\\ \\
In principle this is of great help, since the currents we are calculating, namely $V_{\mu}$, are partially conserved, so we are not required to include any matching factors. However in practice, this is complicated by the fact that the conserved current in the lattice theory is often computationally difficult or impossible to compute. For example, in NRQCD, the partially conserved current corresponding to $SU(N)_V$ is an infinite sum in $1/m_b$ where $m_b$ is the bottom mass. The corresponding current in HISQ is also the sum of a large number of operators. This can be interpreted as a mixing in the renormalization:
\begin{align}
	\langle \mathcal{O}_i \rangle^{\overline{MS}} = \mathcal{Z}_{i,j} \langle \mathcal{O}_j \rangle^{\text{lat}}
\end{align}
In practice, lattice calculations often use only the dominant operators that contribute to the conserved current. Since these will be "close" to the conserved current, one can expect the matching factor to only be a small deviation from unity, and the more sub-dominant operators you add, the overall matching factor should tend towards one.


\subsection{Non-perturbative Renormalization of HISQ Currents}
\subsection{Matching NRQCD currents to $\overline{MS}$}